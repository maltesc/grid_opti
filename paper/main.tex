% !BIB TS-program = biber

\documentclass[a4paper,12pt]{scrartcl}

% Geometry
\usepackage{geometry}
\geometry{a4paper, left=3cm, right=2cm, bottom=4cm}

% Doc Data
\newcommand{\keywords}{Optimization, Modelling, Electrical Grids}

\author{Malte Scharf}
\date{\today}
\title{Modelling and Optimization Abschlussprojekt}
\subtitle{Investitionsoptimierung f\"ur zwei gekoppelte Inselsysteme}

\usepackage[pdftex,
            pdfauthor={Malte Scharf},
            pdftitle={Modelling and Optimization},
            pdfsubject={},
            pdfkeywords={\keywords},
            pdfproducer={Latex with hyperref},
            pdfcreator={pdflatex}]{hyperref}


% Wasserzeichen
\usepackage{draftwatermark}
\SetWatermarkText{Draft}
\SetWatermarkScale{5}
\SetWatermarkColor[gray]{0.9}

% Language and Font
\usepackage[utf8]{inputenc}
\usepackage[ngerman]{babel}

% Space and Graphics
\usepackage{setspace}
\onehalfspacing
\usepackage{graphicx}

% Math
\usepackage{amsfonts} 
\usepackage{amsmath}

% Landscape Format
\usepackage{pdflscape}
% PDF Pages
\usepackage{pdfpages}

% Frames
\usepackage{framed}
\definecolor{shadecolor}{rgb}{0.5,0.5,0.5}

% Literature
\usepackage[
backend=biber,
citestyle=authoryear-icomp,    	% Zitierstil
style=chicago-authordate,
isbn=false,                % ISBN nicht anzeigen, gleiches geht mit nahezu allen anderen Feldern
%ibidtracker=true,
%idemtracker=true,
% ibidpage=true,
% pagetracker=true,          % ebd. bei wiederholten Angaben (false=ausgeschaltet, page=Seite, spread=Doppelseite, true=automatisch)
maxbibnames=50,            % maximale Namen, die im Literaturverzeichnis angezeigt werden (ich wollte alle)
maxcitenames=2,            % maximale Namen, die im Text angezeigt werden, ab 4 wird u.a. nach den ersten Autor angezeigt
autocite=inline,           % regelt Aussehen für \autocite (inline=\parancite)
block=space,               % kleiner horizontaler Platz zwischen den Feldern
backref=false,             % Seiten anzeigen, auf denen die Referenz vorkommt
backrefstyle=three+,       % fasst Seiten zusammen, z.B. S. 2f, 6ff, 7-10
date=short,                % Datumsformat
natbib=true
]{biblatex}

\setlength{\bibitemsep}{1em}     % Abstand zwischen den Literaturangaben
\setlength{\bibhang}{0em}        % Einzug nach jeweils erster Zeile
\bibliography{literature/literature.bib}  % Bibtex-Datei wird schon in der Preambel eingebunden
\ExecuteBibliographyOptions{sorting=nty}
\DefineBibliographyStrings{ngerman}{ 
   andothers = {{et\,al\adddot}},             
} 

% Quotes
\usepackage{csquotes}

% Tables
\usepackage{longtable}
\usepackage{array}


%eigene Commands
% !TEX root = ../main.tex

% Nothing in here



% Akronyme--------
\usepackage[printonlyused]{acronym}
% !TEX root = ../main.tex

\newacro{IKT}{Informations- und Kommunikationstechnologie}




\makeindex

%%% BEGIN DOCUMENT
\begin{document}

\maketitle
%\thispagestyle{empty}

%% !TEX root = ../main.tex


\chapter*{Executive Summary}

\textit{
Content
}


\tableofcontents


% !TEX root = ../main.tex

\section{Introduction} \label{str:intro}

Hello

\section{Model} \label{str:model}

Für Für



% !TEX root = main.tex
\begingroup
\setstretch{0.8}
\setlength\bibitemsep{10pt}

%\addtocounter{chapter}{1}
%\addchap{References}
\printbibliography[title={References}, heading=bibintoc]
%\printbibliography[title={Online-Referenzen}, heading=subbibliography, type=online]
\endgroup

%#Beginn Anhang 
%\chapter*{Appendix} \label{str:appendix}
%% !TEX root = ../main.tex

\section*{Category System} \label{app:category_system}
\todo{Kategoriensystem einfügen}



%% !TEX root = main.tex
\addchap{Eigenständigkeitserklärung}

Ich erkläre, dass ich die vorliegende Arbeit mit dem Titel "`{\textit{\titel - \untertitel}"' 
selbstständig und nur unter Verwendung der angegebenen Hilfsmittel und Quellen angefertigt habe.
Die eingereichte Arbeit ist nicht anderweitig als Prüfungsleistung verwendet worden oder in deutscher oder einer anderen Sprache als Veröffentlichung erschienen.
\\
\\
Seitens des Verfassers bestehen Einwände, die vorliegende Arbeit für die öffentliche Benutzung zur Verfügung zu stellen.
\\
\\

\includegraphics[width=0.3\textwidth]{formales/Unterschrift.jpg}

{\ort}, {\datum}



\end{document}
